%
% This file is part of Cards Against Humanity Fullformat German.
%
% Cards Against Humanity Fullformat German is free software:
% you can redistribute it and/or modify it under the terms of the
% GNU General Public License as published by the Free Software Foundation,
% either version 3 of the License, or (at your option) any later version.
%
% Cards Against Humanity Fullformat German is distributed in the hope that it
% will be useful, but WITHOUT ANY WARRANTY; without even the implied warranty of
% MERCHANTABILITY or FITNESS FOR A PARTICULAR PURPOSE.  See the
% GNU General Public License for more details.
%
% You should have received a copy of the GNU General Public License
% along with Cards Against Humanity Fullformat German.
% If not, see <http://www.gnu.org/licenses/>.
%
%
% Please note: All whitespace characters might break this document. Please
% avoid placing whitespace character that are not intended inside these files.
%
\pgfmathsetmacro{\cardwidth}{6.2}
\pgfmathsetmacro{\cardheight}{8.8}
\pgfmathsetmacro{\stripwidth}{0.6}
\pgfmathsetmacro{\strippadding}{0.1}
\pgfmathsetmacro{\textpadding}{0.1}
\pgfmathsetmacro{\ruleheight}{0.15}
\pgfmathsetmacro{\cardsymbolwidth}{0.25}
\pgfmathsetmacro{\cardsymbolheight}{0.4}
\pgfmathsetmacro{\circlesize}{8}
\pgfmathsetmacro{\paddingtop}{0.8}
\pgfmathsetmacro{\titleheight}{1}
\pgfmathsetmacro{\cardsymbolwidthbig}{1.5}
\pgfmathsetmacro{\cardsymbolheightbig}{.95}

\newcommand{\titleleft}{CARDS AGAINST HUMANITY}
\newcommand{\titleback}{CARDS AGAINST HUMANITY}
\newcommand{\underl}{\rule{2cm}{.5pt}\hspace{3pt}}

%%%%%%%%%%%%%%%%%%%%%%%%%%%%%%%%

\makeatletter
\newcommand\modulo[2]{\@tempcnta=#1
        \divide\@tempcnta by #2
        \multiply\@tempcnta by #2
        \multiply\@tempcnta by -1
        \advance\@tempcnta by #1\relax
        \the\@tempcnta}
\makeatother

%%%%%%%%%%%%%%%%%%%%%%%%%%%%%%%%

\newcounter{CntTitle}
\newcommand\titlecard[1]{%
	\stepcounter{CntTitle}%
	\csdef{title\theCntTitle}{#1}%
}
\newcommand\getTitle[1]{\csuse{title#1}}

\newcounter{CntWhite}
\newcommand\whitecard[2]{%
	\stepcounter{CntWhite}%
	\csdef{whiteTopic\theCntWhite}{#1}%
	\csdef{whiteTitle\theCntWhite}{#2}%
}
\newcommand\getWhiteTopic[1]{\csuse{whiteTopic#1}}
\newcommand\getWhiteTitle[1]{\csuse{whiteTitle#1}}

\newcounter{CntBlack}
\newcommand\blackcard[3]{%
	\stepcounter{CntBlack}%
	\csdef{blackCount\theCntBlack}{#1}%
	\csdef{blackTopic\theCntBlack}{#2}%
	\csdef{blackTitle\theCntBlack}{#3}%
}
\newcommand\getBlackCount[1]{\csuse{blackCount#1}}
\newcommand\getBlackTopic[1]{\csuse{blackTopic#1}}
\newcommand\getBlackTitle[1]{\csuse{blackTitle#1}}

%%%%%%%%%%%%%%%%%%%%%%%%%%%%%%%%

\newcounter{CntList}
\newcommand\makeCardLists{%
	% Title Cards %
	\setcounter{CntList}{0}%
	\whiledo {\value{CntList} < \value{CntTitle}}%
		{%
			\foreach \n in {0,...,8}%
				{%
					\ifnum\value{CntTitle}>\number\numexpr\value{CntList}+\n%
						\drawTitlecard{\getTitle{\number\numexpr\value{CntList}+\n+1}}%
					\fi%
				}%
			\newpage
			\foreach \n in {2,...,0}%
				{%
					\ifnum\value{CntTitle}>\number\numexpr\value{CntList}+\n%
						\drawTitlecard{\getTitle{\number\numexpr\value{CntList}+\n+1}}%
					\fi%
				}%
			\foreach \n in {5,...,3}%
				{%
					\ifnum\value{CntTitle}>\number\numexpr\value{CntList}+\n%
						\drawTitlecard{\getTitle{\number\numexpr\value{CntList}+\n+1}}%
					\fi%
				}%
			\foreach \n in {8,...,6}%
				{%
					\ifnum\value{CntTitle}>\number\numexpr\value{CntList}+\n%
						\drawTitlecard{\getTitle{\number\numexpr\value{CntList}+\n+1}}%
					\fi%
				}%
			\newpage
			\addtocounter{CntList}{9}%
		}%
	%%%%%%%%%%%%%%%%%%%%%%%%%%%%
	% White Cards %
	\setcounter{CntList}{0}%
	\whiledo {\value{CntList} < \value{CntWhite}}%
		{%
			\stepcounter{CntList}%
			\drawWhitecard{\getWhiteTopic{\theCntList}}{\getWhiteTitle{\theCntList}}%
			%
			\ifnum\value{CntList}=\number\numexpr\value{CntList}/9*9\relax%
				\newpage%
				\drawWhitecardback\drawWhitecardback\drawWhitecardback%
				\drawWhitecardback\drawWhitecardback\drawWhitecardback%
				\drawWhitecardback\drawWhitecardback\drawWhitecardback%
				\newpage%
			\else%
				\ifnum\value{CntList}=\value{CntWhite}%
					\newpage%
					\pgfmathtruncatemacro{\remainingCards}{\value{CntList}-div(\value{CntList},9)*9}
					\foreach \n in {1,...,\remainingCards}%
						{\drawWhitecardback}%
					\newpage%
				\fi%
			\fi%
		}%
	%%%%%%%%%%%%%%%%%%%%%%%%%%%%
	% Black Cards %
	\setcounter{CntList}{0}%
	\whiledo {\value{CntList} < \value{CntBlack}}%
		{%
			\stepcounter{CntList}%
			\drawBlackcard{\getBlackCount{\theCntList}}{\getBlackTopic{\theCntList}}{\getBlackTitle{\theCntList}}%
			%
			\ifnum\value{CntList}=\number\numexpr\value{CntList}/9*9\relax%
				\newpage%
				\drawBlackcardback\drawBlackcardback\drawBlackcardback%
				\drawBlackcardback\drawBlackcardback\drawBlackcardback%
				\drawBlackcardback\drawBlackcardback\drawBlackcardback%
				\newpage%
			\else%
				\ifnum\value{CntList}=\value{CntBlack}%
					\newpage%
					\pgfmathtruncatemacro{\remainingCards}{\value{CntList}-div(\value{CntList},9)*9}
					\foreach \n in {1,...,\remainingCards}%
						{\drawBlackcardback}%
					\newpage%
				\fi%
			\fi%
		}%
}

%%%%%%%%%%%%%%%%%%%%%%%%%%%%%%%%

\newcommand{\drawBlackcard}[3] {%
	\begin{tikzpicture}%
		\draw (0,0) rectangle (\cardwidth cm,\cardheight cm);
		% Bar left
		\fill[black] (0, 0) rectangle (\stripwidth,\cardheight) node[rotate=90,above left,white,font=\large] {\titleleft};
		% Icon
		\node[fill=black, draw=white, minimum height = \cardsymbolwidth cm, minimum width = \cardsymbolheight cm, anchor=east, rotate=15] at (4*\stripwidth/5, \stripwidth/2) { } ;
		\node[fill=white, draw=black, minimum height = \cardsymbolwidth cm, minimum width = \cardsymbolheight cm, anchor=east, rotate=0] at (4*\stripwidth/5, \stripwidth/2) { } ;
		\node[fill=white, draw=black, minimum height = \cardsymbolwidth cm, minimum width = \cardsymbolheight cm, anchor=east, rotate=-20] at (4*\stripwidth/5, \stripwidth/2) { } ;
		% Text
		\node[text width=(\cardwidth-\strippadding-\stripwidth-2*\textpadding-0.3)*1cm,below right, anchor=north west] at (\strippadding+\stripwidth+\textpadding,\cardheight-\textpadding-\paddingtop)
		{{\flushleft \Large #3\par}};
		\fill[black, anchor=south west](\stripwidth cm + \circlesize*1.5 pt, \circlesize*2 pt) circle(\circlesize pt);
		\node[anchor=center, color=white] at (\stripwidth cm + \circlesize*1.5 pt, \circlesize*2 pt) {\textbf{#1}};
		\node[anchor=west, color=black] at (\stripwidth cm + \circlesize*2.5 pt, \circlesize*2 pt)
		{\ifnum#1=1
			Karte
		\else
			Karten
		\fi};
		% Category
		\node[anchor=south east] at (\cardwidth, 0) {\scriptsize #2};
	\end{tikzpicture}%
	\hspace{0pt}%
}%

\newcommand{\drawWhitecard}[2] {%
	\begin{tikzpicture}%
		\draw (0,0) rectangle (\cardwidth cm,\cardheight cm);
		% Bar left
		\draw[gray] (0,0) rectangle (\stripwidth,\cardheight) node[rotate=90,above left,black,font=\large] {\titleleft};
		% Icon
		\node[fill=black, draw=black, minimum height = \cardsymbolwidth cm, minimum width = \cardsymbolheight cm, anchor=east, rotate=15] at (4*\stripwidth/5, \stripwidth/2) { } ;
		\node[fill=white, draw=black, minimum height = \cardsymbolwidth cm, minimum width = \cardsymbolheight cm, anchor=east, rotate=0] at (4*\stripwidth/5, \stripwidth/2) { } ;
		\node[fill=white, draw=black, minimum height = \cardsymbolwidth cm, minimum width = \cardsymbolheight cm, anchor=east, rotate=-20] at (4*\stripwidth/5, \stripwidth/2) { } ;
		% Text
		\node[text width=(\cardwidth-\strippadding-\stripwidth-2*\textpadding-0.3)*1cm,below right, anchor=north west] at (\strippadding+\stripwidth+\textpadding,\cardheight-\textpadding-\paddingtop)
		{{\flushleft \LARGE #2\par}};
		% Category
		\node[anchor=south east] at (\cardwidth, 0) {#1};
	\end{tikzpicture}%
	\hspace{0pt}%
}%


\newcommand{\drawTitlecard}[1] {%
	\begin{tikzpicture}%
		\draw (0,0) rectangle (\cardwidth cm,\cardheight cm);
		% Bar middle
		\draw[black] (0,\cardheight/2 - \titleheight) rectangle (\cardwidth,\stripwidth*3/4 + \cardheight/2 - \titleheight);
		\node[anchor = south , black,font=\scriptsize] at (\cardwidth/2,\cardheight/2 - \titleheight - \stripwidth/20) {\titleleft};
		%categorytitle
		\fill[black](0,\cardheight/2 + \titleheight) rectangle (\cardwidth, \stripwidth*3/4 + \cardheight/2 - \titleheight);
		\node[anchor = center, white,font=\Huge] at (\cardwidth/2,\cardheight/2 + \titleheight*0.2) {#1};
		% Icon
		\node[fill=black, draw=black, minimum height = \cardsymbolwidthbig cm, minimum width = \cardsymbolheightbig cm, anchor=south, rotate=15] at (\cardwidth/2, 1cm) { } ;
		\node[fill=white, draw=black, minimum height = \cardsymbolwidthbig cm, minimum width = \cardsymbolheightbig cm, anchor=south, rotate=0] at (\cardwidth/2, 1cm) { } ;
		\node[fill=white, draw=black, minimum height = \cardsymbolwidthbig cm, minimum width = \cardsymbolheightbig cm, anchor=south, rotate=-20] at (\cardwidth/2, 1cm) { } ;
		% Text
	\end{tikzpicture}%
	\hspace{0pt}%
}%

\newcommand{\drawWhitecardback} {%
	\begin{tikzpicture}%
		\draw (0,0) rectangle (\cardwidth cm,\cardheight cm);
		% Title
		\node[anchor=center,text width=(\cardwidth-\strippadding-\stripwidth-2*\textpadding-0.3)*1cm,below right] at (\strippadding+\stripwidth+\textpadding,\cardheight-\textpadding-\paddingtop)
		{{\flushleft \Huge \titleback\par}};
		% Icon
		\node[fill=black, draw=black, minimum height = \cardsymbolwidthbig cm, minimum width = \cardsymbolheightbig cm, anchor=south, rotate=15] at (\cardwidth/2, 1cm) { } ;
		\node[fill=white, draw=black, minimum height = \cardsymbolwidthbig cm, minimum width = \cardsymbolheightbig cm, anchor=south, rotate=0] at (\cardwidth/2, 1cm) { } ;
		\node[fill=white, draw=black, minimum height = \cardsymbolwidthbig cm, minimum width = \cardsymbolheightbig cm, anchor=south, rotate=-20] at (\cardwidth/2, 1cm) { } ;
		% Text
	\end{tikzpicture}%
	\hspace{0pt}%
}%

\newcommand{\drawBlackcardback} {%
	\begin{tikzpicture}%
		\draw (0,0) rectangle (\cardwidth cm,\cardheight cm);
		% Bar left
		\fill[black] (0, 0) rectangle (\stripwidth,\cardheight) node[rotate=90,above left,white,font=\large] {};
		% Title
		\node[anchor=center,black,text width=(\cardwidth-\strippadding-\stripwidth-2*\textpadding-0.3)*1cm,below right] at (\strippadding+\stripwidth+\textpadding,\cardheight-\textpadding-\paddingtop)
		{{\flushleft \Huge \titleback\par}};
		% Icon
		\node[fill=black, draw=black, minimum height = \cardsymbolwidthbig cm, minimum width = \cardsymbolheightbig cm, anchor=south, rotate=15] at (\cardwidth/2, 1cm) { } ;
		\node[fill=white, draw=black, minimum height = \cardsymbolwidthbig cm, minimum width = \cardsymbolheightbig cm, anchor=south, rotate=0] at (\cardwidth/2, 1cm) { } ;
		\node[fill=white, draw=black, minimum height = \cardsymbolwidthbig cm, minimum width = \cardsymbolheightbig cm, anchor=south, rotate=-20] at (\cardwidth/2, 1cm) { } ;
		% Text
	\end{tikzpicture}%
	\hspace{0pt}%
}%

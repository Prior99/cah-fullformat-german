\documentclass[a4paper,parskip]{scrartcl}
\usepackage[top=10mm,bottom=10mm,left=15mm,right=15mm]{geometry}
\usepackage{tikz}
%\usepackage{pifont}
%\usepackage{anttor}
\usepackage[sfdefault]{ClearSans}
\usepackage[utf8]{inputenc}
\begin{document}
\pgfmathsetmacro{\cardwidth}{6.2}
\pgfmathsetmacro{\cardheight}{8.8}
\pgfmathsetmacro{\stripwidth}{0.6}
\pgfmathsetmacro{\strippadding}{0.1}
\pgfmathsetmacro{\textpadding}{0.1}
\pgfmathsetmacro{\ruleheight}{0.15}
\pgfmathsetmacro{\cardsymbolwidth}{0.25}
\pgfmathsetmacro{\cardsymbolheight}{0.4}

\setlength{\baselineskip}{16pt}

\newcommand{\titleleft}{CARDS AGAINST HUMANITY}
\newcommand{\underl}{\rule{2cm}{.5pt}}

\newcommand{\blackcard}[2] {\begin{tikzpicture}%
    \draw (0,0) rectangle (\cardwidth cm,\cardheight cm);
        \fill[black] (0, 0) rectangle (\stripwidth,\cardheight) node[rotate=90,above left,white,font=\large] {
            \titleleft
        };
        \node[fill=black, draw=white, minimum height = \cardsymbolwidth cm, minimum width = \cardsymbolheight cm, anchor=east, rotate=15] at (4*\stripwidth/5, \stripwidth/2) { } ;
        \node[fill=white, draw=black, minimum height = \cardsymbolwidth cm, minimum width = \cardsymbolheight cm, anchor=east, rotate=0] at (4*\stripwidth/5, \stripwidth/2) { } ;
        \node[fill=white, draw=black, minimum height = \cardsymbolwidth cm, minimum width = \cardsymbolheight cm, anchor=east, rotate=-20] at (4*\stripwidth/5, \stripwidth/2) { } ;

        \node[text width=(\cardwidth-\strippadding-\stripwidth-2*\textpadding-0.3)*1cm,below right, anchor=west] at (\strippadding+\stripwidth+\textpadding,\cardheight*1/2-\textpadding) {
            {\huge #2\par}
        };
        \node[anchor=south east] at (\cardwidth, 0) {\scriptsize #1};
\end{tikzpicture}}
\newcommand{\whitecard}[2] {\begin{tikzpicture}%
    \draw (0,0) rectangle (\cardwidth cm,\cardheight cm);
        \draw[gray] (0,0) rectangle (0\stripwidth,\cardheight0) node[rotate=90,above left,black,font=\large] {\titleleft};
        \node[fill=black, draw=black, minimum height = \cardsymbolwidth cm, minimum width = \cardsymbolheight cm, anchor=east, rotate=15] at (4*\stripwidth/5, \stripwidth/2) { } ;
        \node[fill=white, draw=black, minimum height = \cardsymbolwidth cm, minimum width = \cardsymbolheight cm, anchor=east, rotate=0] at (4*\stripwidth/5, \stripwidth/2) { } ;
        \node[fill=white, draw=black, minimum height = \cardsymbolwidth cm, minimum width = \cardsymbolheight cm, anchor=east, rotate=-20] at (4*\stripwidth/5, \stripwidth/2) { } ;

        \node[text width=(\cardwidth-\strippadding-\stripwidth-2*\textpadding-0.3)*1cm,below right, anchor=west] at (\strippadding+\stripwidth+\textpadding,\cardheight*2/3-\textpadding) {
            {\huge #2\par}
        };
        \node[anchor=south east] at (\cardwidth, 0) {\scriptsize #1};
\end{tikzpicture}}

\whitecard{Basis}{In Flamme stehen.}%
\whitecard{Basis}{Rassismus.}%
\whitecard{Basis}{Den Geruch alter Leute}
\whitecard{Basis}{In Flamme stehen.}%
\whitecard{Basis}{Rassismus.}%
\whitecard{FSK16}{Einen Mikropenis}
\whitecard{Basis}{Frauen in Joghurt Werbespots.}%
\whitecard{Basis}{Sich einen Scheiss um die dritte Welt scheren.}%
\whitecard{FSK18}{Ein Einweckglas in meinen Anus einführen.}
\whitecard{Basis}{Gerichtlich angeordneter Entzug.}%
\whitecard{Böse}{Eine Windmühle voller Leichen.}%
\whitecard{Basis}{Die Schwulen.}
\whitecard{Basis}{Ein übergroßer Lolli.}%
\whitecard{Basis}{Afrikanische Kinder.}%
\whitecard{Basis}{Eine asymmetrische Brustvergößerung.}
\blackcard{Basis}{Ich habe \underl nie wirklich verstanden, bis ich auf \underl gestoßen bin.}%
\blackcard{FSK18}{}%
\blackcard{Basis}{Was ist die neue Modediät?}%
\blackcard{Basis}{Was hat nie verfehlt, eine Party zu beleben?}
\blackcard{Basis}{\underl + \underl = \underl.}%
\blackcard{Drogen}{Was ist meine Anti-Droge?%
\blackcard{Basis}{Ja richtig, ich habe \underl getötet. Wie fragst du? \underl.}
\blackcard{Drogen}{Als einen Acid-trip hatte, \underl wurde zu \underl.}%
\blackcard{Basis}{Wenn ich ein Milliadär wäre, um wem zu gedenken würde ich eine Statue bauen?}%
\blackcard{Basis}{Wenn ich König von Deutschland wär, würd ich das \underl Ministerium ins Leben rufen.}
\blackcard{Basic}{Dreamworks präsentiert "\underl die Geschichte von \underl."}%
\blackcard{}{}%
\blackcard{}{}
\blackcard{}{}%
\blackcard{}{}%
\blackcard{}{}
\blackcard{}{}%
\blackcard{}{}%
\blackcard{}{}
\blackcard{}{}%
\blackcard{}{}%
\blackcard{}{}
\blackcard{}{}%
\blackcard{}{}%
\blackcard{}{}
\blackcard{}{}%
\blackcard{}{}%
\blackcard{}{}
\blackcard{}{}%
\blackcard{}{}%
\blackcard{}{}
\blackcard{}{}%
\blackcard{}{}%
\blackcard{}{}

\end{document}
